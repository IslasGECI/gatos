\section{Introducción}
%%%%%%%%%%%%%%%%%%%%%% Método    %%%%%%%%%%%%%%%%%%%%%%%%%%%%%%%%%%%%%%%%%%%%%%
\section{M\'etodo}
\begin{frame}
\begin{itemize}
  \item Estimaci\'on de la densidad a partir de la tasa de trampeo
  \item Estimaci\'on de la varianza
  \item Pruebas de campo
  \item Par\'ametros de las c\'amaras
  \item Par\'ametros de los animales
  \item Censo animal
  \item Simulaci\'on de la precisi\'on
\end{itemize}
\end{frame}
\begin{frame}
\begin{figure}[h]
\centering
\includegraphics[width=80mm]{../referencias/imagenes/figura1_rowcliffe.png}
\caption{Diagrama que muestra la variaci\'on en la secci\'on presentada a
animales por la zona de detecci\'on de la trampas-c\'amara. Las direcciones de
acercamiento de los animales son indicadas por flechas, la zona de detecci\'on
es el \'area sombreada, definida por una distancia $r$ y un \'angulo $\theta$, y
la barra obscura es la secci\'on presentada.}
\end{figure}
\end{frame}
%%%%
\begin{frame}
\frametitle{Ecuaciones}
$$y=2rtvD$$
$$\frac{2\int\limits^{\pi/2}_{(\pi-\theta)/2}2r\sin\left(\frac{\theta}{2}\right)
\sin \left(\gamma\right)\mbox{d}\gamma+2\int\limits^{\pi/2}_{\theta}r
\sin\left(\gamma\right)\mbox{d}\gamma}{\pi}=r\frac{2+\theta}{\pi}$$
$$y=\frac{2+\theta}{\pi}rtvD$$
$$D=\frac{y}{t}\frac{\pi}{vr(2+\theta)}$$
\end{frame}
%%%%%%%%%%%%%%%%%%%%
\begin{frame}
\begin{figure}[h]
\centering
\includegraphics[width=60mm]{../referencias/imagenes/tabla1_rowcliffe.png}
\caption{Resumen de los datos de censo y trampas-c\'amara dividido por especies
y zonas.}
\label{tabla1}
\end{figure}
\end{frame}
%%%%%%%%%%%%%%%%%%%%  Resultados %%%%%%%%%%%%%%%%%%%%%%%%%%%%%%%%%%%%%%%%%%%%%%%
\section{Resultados}
\begin{frame}
\begin{itemize}
  \item Resultados de campo
  \item Resultados de la simulaci\'on
\end{itemize}
\end{frame}
%%%
\begin{frame}
\begin{figure}[h]
\centering
\includegraphics[width=70mm]{../referencias/imagenes/figura2a_rowcliffe.png}
\caption{Tasa de trampas-c\'amaras contra densidades por censo. La l\'inea
punteada indica la regresi\'on lineal. Los puntos est\'an agrupados por especies
y para cada especie se hicieron censos en las cuatro \'areas de estudio.}
\label{figura2a}
\end{figure}
\end{frame}
%%%%%%%%%%%%%%%%%%%%
\begin{frame}
\begin{figure}[h]
\centering
\includegraphics[width=80mm]{../referencias/imagenes/figura2b_rowcliffe.png}
\caption{Densidad estimada de las cuatro especies contra las densidades por
censo. La l\'inea diagional representa la identidad.}
\label{figura2b}
\end{figure}
\end{frame}
%%%%%%%%%%%%%%%%%%%%
\begin{frame}
\begin{figure}[h]
\centering
\includegraphics[width=110mm]{../referencias/imagenes/figura3_rowcliffe.pdf}
\caption{El conjunto de las cuatro densidades, comparado entre las densidades
por censo y las densidades estimadas mediante las trampas-c\'amara.}
\label{figura3}
\end{figure}
\end{frame}
%%%%%%%%%%%%%%%%%%%%
\begin{frame}
\begin{figure}[h]
\centering
\includegraphics[width=100mm]{../referencias/imagenes/tabla2_rowcliffe.png}
\caption{Resumen de las varibles independientes requeridas para calcular la
densidad animal a partir de las tasas de trampas-c\'amara.}
\label{tabla2}
\end{figure}
\end{frame}
%%%%%%%%%%%%%%%%%%%%
\begin{frame}
\begin{figure}[h]
\centering
\includegraphics[width=100mm]{../referencias/imagenes/figura4a_rowcliffe.pdf}
\caption{
La precisi\'on de la densidad estimada de datos simulados en relaci\'on
a la variaci\'on del esfuerzo de muestreo, suponiendo alta y baja varianza en
la tasa de trampas-c\'amara (l\'inea superior e inferior, respectivamente).
Cambia el n\'umero de c\'amaras mientras se mantiene constante el tiempo por
c\'amara.}
\label{figura4a}
\end{figure}
\end{frame}
%%%%%%%%%%%%%%%%%%%%
\begin{frame}
\begin{figure}[h]
\centering
\includegraphics[width=50mm]{../referencias/imagenes/figura4bc_rowcliffe.pdf}
\caption{
La precisi\'on de la densidad estimada de datos simulados en relaci\'on
a la variaci\'on del esfuerzo de muestreo, suponiendo alta y baja varianza en
la tasa de trampas-c\'amara (l\'inea superior e inferior, respectivamente).
(b) El tiempo por c\'amara cambia (n\'umero total de fotograf\'ias) y el
n\'umero total de trampas-c\'amara es constante. (c) El n\'umero de ubicaciones
de las trampas-c\'amara var\'ia pero el tiempo total es constante.}
\label{figura4b}
\end{figure}
\end{frame}
%%%%%%%%%%%%%%%%%%%%
\begin{frame}
\begin{figure}[h]
\centering
\includegraphics[width=50mm]{../referencias/imagenes/figura5_rowcliffe.pdf}
\caption{Esfuerzo de trampeo esperado (d\'ias trampa indicados por contornos)
requiere lograr 10 fotograf\'ias dada una variaci\'on de densidad y suponiendo
un grupo de tama\~no 1. Las combinaci\'on t\'ipica de recorridos por d\'ia y
densidad son indicados para carn\'ivoros (C), ungulados (U) y roedores (R),
calculadas utilizando ecuaciones alom\'etricas para el recorrido por d\'ia y la
densidad a una capacidad de carga e ilustrando densidades entre 10\% y 100\% de
la capacidad de carga. }
\label{figura5}
\end{figure}
\end{frame}
%%%%%%%%%%%%%%%%%%%% Discusión %%%%%%%%%%%%%%%%%%%%%%%%%%%%%%%%%%%%%%%%%%%%%%%%%
\begin{frame}
\bibliography{referencias} % Despliega la bibliografía que se encuentra en el archivo  `referencias.bib`
\end{frame}
