\documentclass{article}
\usepackage[utf8]{inputenc}
\usepackage[spanish]{babel}
\usepackage{hyperref}
\usepackage{pythontex}
\usepackage{natbib}
\bibliographystyle{estilo_referencias}

\author{Braulio Rojas y Daniela Mungu\'ia}
\title{Cabras en Guadalupe}

\renewcommand{\familydefault}{\sfdefault}

\begin{document}
\maketitle

% Comienza el ambiente de python
\begin{pycode}
import datapackage
directorioDataPackage = '../inst/extdata/datapackage.json'   #directorio del datapackage
PaqueteDatos = datapackage.DataPackage(directorioDataPackage)#paquete con datos y metadatos
Recursos = PaqueteDatos.descriptor['resource']               #estructura que contiene los campos dentro de resource
nRecursos = len(Recursos[0]['schema']['fields'])             #número de campos dentro de la estructura recursos
recursosJuntos=[];                                           #vector a rellenar en el ciclo
for iRecursos in range(0,nRecursos):
    recursosJuntos.append((Recursos[0]["schema"]["fields"][iRecursos]["name"])) #rellena vector con los nombres de los campos dentro de recursos

\end{pycode}


\section*{Resumen} % El resumen menciona la relación de este análisis con la
% erradicación de gatos. Se menciona también el estado actual del análisis de 
% cabras, el paso inmediato siguiente y el estado final.
En este reporte se hace una descripci\'on de los datos de la erradicaci\'on 
de cabras en Isla Guadalupe, la cual se limita a una menci\'on de las variables 
que contiene el archivo \textit{\py{PaqueteDatos.descriptor['name']}.csv}. Lo 
siguiente que se debe hacer con los datos es una regresi\'on lineal (blancos 
acumulados en funci\'on de las horas de vuelo acumuladas y despu\'es con la 
derivada de eso). Al terminar, se tendr\'a la estimación del n\'umero inicial 
de cabras utilizando el m\'etodo tradicional y se podr\'a evaluar este resultado 
contra el n\'umero real de cabras erradicadas. El resultado final servir\'a para 
evaluar y calibrar la estimaci\'on del tamaño de poblaci\'on de gatos, la cual 
se pretende calcular con el método \textit{Catch-Effort} presentado en \cite{krebs2012}.


\section*{Metodolog\'ia} %Inicia sección de metodología
\subsection*{Datos} % Descripción de los datos
Se hizo una revisi\'on de los datos contenido en el archivo
\textit{\py{PaqueteDatos.descriptor['name']}.csv}. Dicho archivo contiene
informaci\'on que fue guardada en un Tabular Data Package (TDP), el cual es un
paquete que contiene una colecci\'on de datos en un archivo CSV (del ingl\'es,
Comma-Separated Values) y la descripci\'on de los mismos guardados en un JSON
(del ingl\'es, JavaScript Object Notation) \citep{Walsh2012}.
El TDP creado contiene el archivo
\textit{\py{PaqueteDatos.descriptor['name']}.csv} con datos y el correspondiente
\textit{datapackage.json} con metadatos. Es importante mencionar que, para que
se cumpla con la nomenclatura de los TDP el archivo de metadatos debe llamarse
\textit{datapackage}.

La informaci\'on contenida en el archivo de datos proviene de
\py{PaqueteDatos.descriptor['source']} en Isla Guadalupe. En el archivo las
variables son: \py{', '.join(recursosJuntos[0:nRecursos-1])} y
\py{recursosJuntos[nRecursos-1]}.


\section*{Referencias}

\bibliography{referencias}
\end{document}
