\documentclass{article}
\usepackage{graphicx}
\usepackage{color}
\usepackage[latin1]{inputenc}
\usepackage{csvsimple}
\usepackage{pythontex}
\usepackage{natbib}
\usepackage[spanish]{babel}
\bibliographystyle{estilo_referencias}
\renewcommand{\tablename}{Tabla}
\renewcommand{\figurename}{Figura}
 \renewcommand{\refname}{Referencias}
\sloppy
\setlength{\textwidth}{160mm}
\setlength{\oddsidemargin}{-5mm}
\setlength{\evensidemargin}{-5mm}
\definecolor{lightgray}{gray}{0.5}
\setlength{\parindent}{0pt}

\title{Estimaci\'on bayesiana del tama\~no de la poblaci\'on}
\author{ Braulio Rojas \\ Direcci\'on de An\'alisis de Cient\'ifico}
\begin{document}
\maketitle
%%%%%%%%%%%%%%%%%%%%%%%%%%%%%%%%%%%%%%%%%%%%%%%%%%%%%%%%%%%%%%%%%%%%%%%%%%%%%%%%
% En esta sección se leen los archivos que se utilizarán por `pythontex`. Son
% dos archivos los que se utilizan: `datos-objetivo.json` y `datos_histograma_q.csv`. En el primer
% archivo vienen los datos sintéticos que son la "adivinanza" para el programa.
% El segundo archivo se utiliza para hacer la descripci\'on de la primera
% gráfica.
% Esta sección sigue los siguientes pasos:
% 1.-Define las direcciones de los archivos ( *.json y *.csv);
% 2.-Importa las módulos que servirán para la lectura de los archivos;
% 3.-Abre el archivo *.json y guarda su información en la variable `datosJson`.
% 4.-Abre el archivo *.csv y:
%  4.1.-Lee todo como texto;
%  4.2.-Define las listas eje_x y eje_y
%  4.3.-Transforma las variables de texto a reales.
\pyc{rutaJson='..\\resultados\\datos-objetivo.json'}
\pyc{rutaCSV='..\\resultados\\datos_histograma_q.csv'}
\begin{pycode}
import json
import csv
with open(rutaJson,encoding='utf-8') as archivoJson:
      datosJson=json.load(archivoJson)
with open(rutaCSV,'r') as archivoCSV:
      texto=csv.reader(archivoCSV,delimiter=',')
      eje_x=[];eje_y=[];
      for row in texto:
        eje_x.append(row[0])
        eje_y.append(row[1])
      nDatos=eje_x.index(eje_x[-1])
      for iDatos in range(1,nDatos+1):
        eje_x[iDatos]=float(eje_x[iDatos])
        eje_y[iDatos]=int(float(eje_y[iDatos]))
\end{pycode}
%%%%%%%%%%%%%%%%%%%%%%%%%%%%%%%%%%%%%%%%%%%%%%%%%%%%%%%%%%%%%%%%%%%%%%%%%%%%%%%%
\subsection*{Planteamiento del problema}
Aunque menos del 20\% de las especies animales en la tierra est\'an restringidas
a las islas, el 75\% de las especies extintas desde 1600 han sido en las islas.
Los gatos ferales han estados asociados con la extinci\'on de especies
end\'emicas en las islas en todo el mundo. La remoci\'on de estos gatos de las
islas es una menera efectiva de proteger la biodiversidad \citep{Wood2002}.

\cite{Ramsey2010} presenta una metodolog\'ia para declarar terminada una
erradicaci\'on. La metodolog\'ia incorpora informaci\'on del costo del
monitoreo y el costo de una declaraci\'on incorrecta de la erradicaci\'on. La
estad\'istica bayesiana es la teor\'ia que sustentan dicha metodolog\'ia. A
continuaci\'on, hacemos una breve revisi\'on de los principios de la
estad\'istica bayesiana y los utilizamos en la estimaci\'on del tama\~no de
poblaci\'on.



% Esta sección corresponde a la introducción del método de estadística
% Bayesiana. Consiste en dos subsecciones: estadística bayesiana y el teorema de
% Bayes.
\subsection*{Estad\'istica Bayesiana}
La {\it estad\'istica convencional} (EC) considera, a partir de una muestra
aleatoria de una poblaci\'on, que es posible estimar el valor verdadero de
alg\'un par\'ametro de la poblaci\'on \citep{Ellison1996}. De manera equivalente,
la {\it estad\'istica bayesiana} (EB) es aquella que considera a las muestras
y los par\'ametros de una distribuci\'on como variables aleatorias (sin valores
fijos) \citep{Gilks1996}.

\cite{McCarthy2007} encuentra dos ventajas de usar la EB
sobre la EC: la EB se puede usar para hacer predicciones
probabil\'isticas del estado del mundo (por ejemplo de la erradicaci\'on de
especies), mientras que la EC nos restringe a promedios de r\'eplicas de muestras
de datos. La segunda ventaja es que la EB nos permite agregar informaci\'on
relevante para el an\'alisis de las funciones distribuciones de probabilidad
(FDP) de los par\'ametros.

Las componentes de conocimiento que utiliza la EB son: conocimiento previo,
nuevos datos, el modelo y el conocimiento nuevo. La incorporaci\'on de las
cuatro componentes se logra mediante el teorema de Bayes \citep{McCarthy2007}.
\subsection*{El teorema de Bayes}
El teorema de Bayes dice lo siguiente:
\begin{equation}\label{ecTBayes}
p(\theta |\vec x) = \frac{p(\vec x |\theta)p(\theta)}{p(\vec x)}.
\end{equation}

La probabiliadad de que una hip\'otesis, $\theta$, sea verdadera dado un
conjunto de datos, $\vec x$, es el producto de la probabilidad de obtener los
datos siendo verdadera la hip\'otesis, $p(\vec x|\theta)$, por la probabilidad
de que la hip\'otesis suceda, $p(\theta)$, entre la probabilidad de que sucedan
los datos sin importar nada m\'as, $p(\vec x)$. Y expresado en t\'erminos de
FDP \citep{Lynch2007} se tiene
\begin{equation}\label{ecTBayesDistribuciones}
f(\theta |\mbox{datos}) = \frac{g(\mbox{datos} |\theta)h(\theta)}{k(\mbox{datos})},
\end{equation}
donde $f(\theta |\mbox{datos})$ es la FDP posterior para los
par\'ametros $\theta$, $h(\theta)$ es la FDP {\it previa} de los
par\'ametros y $k(\mbox{datos})$ es la probabilidad marginal de los datos y
actua como una constante de normalizaci\'on. De lo anterior se tiene que:
\begin{equation}\label{ecTBayesProp}
f(\theta |\mbox{datos}) \propto g(\mbox{datos} |\theta)h(\theta).
\end{equation}

A diferencia de lo que se obtiene con EC para los par\'ametros (un valor puntual
y su error est\'andar), con la EB se obtiene una FDP para los
par\'ametros y entonces se puede utilizar esta FDP para obtener otra
informaci\'on \citep{Lynch2007}.

\subsection*{Muestras de Gibbs}
Si tenemos el vector de las $k$ hip\'otesis, $\theta=(\theta_1,...,\theta_k)$, y
lo que nos interesa obtener es la distribuci\'on conjuta posterior
$f(\theta |\mbox{datos})=f(\theta_1,...,\theta_k|\mbox{datos})$, eso lo podemos
lograr mediante m\'etodos num\'ericos. Una propuesta
para conseguirlo es simular muestras para diferentes valores de las $k$
hip\'otesis. Si esas muestras vienen de distribuciones completamente
condicionadas, los valores simulados podr\'an ser utilizados para calcular la
distribuci\'on posterior de inter\'es \citep{Bernardo2000}.

Las muestras de Gibbs es una t\'ecnica que simula muestras a partir de cadenas
de Markov y es un m\'etodo iterativo. El procedimiento utiliza a la densidadad
completamente condicionada:
$$
f(\theta_j|\vec x,\theta_i);
$$
con $i=1,...,k$ pero $j\neq i$ y a un conjunto arbitrario de valores iniciales
$\theta^{(0)}=(\theta^{(0)}_1,...,\theta^{(0)}_k)$. Entonces, las muestras de
las $k$ hip\'otesis ocurren de la siguiente manera:
$$\theta^{(1)}_{1}\sim f_1(\theta_1|\vec x,\theta^{(0)}_2,...,\theta^{(0)}_k)$$
$$\theta^{(1)}_{2}\sim f_2(\theta_2|\vec x,\theta^{(1)}_1,...,\theta^{(0)}_k)$$
$$\vdots$$
$$\theta^{(1)}_{k}\sim f_k(\theta_k|\vec x,\theta^{(1)}_1,...,\theta^{(1)}_{k-1})$$
$$\theta^{(2)}_{1}\sim f_1(\theta_1|\vec x,\theta^{(1)}_2,...,\theta^{(1)}_k)$$
$$\vdots$$
La forma de las $f_i$ dependen de la selecci\'on de las FDP {\it previas} para
las $\theta_i$ y funci\'on verosimilitud, $g(\mbox{datos} |\theta)$. Una
adecuada selecci\'on permitir\'a que las densidades completamente condicionadas,
$f_i$, tengan una forma conocida.

% En esta sección se presentan los valores objetivos ("adivinanza") y los datos
% sintéticos.
\subsection*{C\'alculo del tama\~no inicial de la poblaci\'on y de la
probabilidad de captura}
Para hacer una primera aproximaci\'on del problema, vamos a suponer que se hacen
capturas con remplazo, lo que pasar\'ia con c\'amaras trampas. Vamos a
seleccionar, de manera aleatoria, valores para el tama\~no inicial de una
poblaci\'on ($N_o$) y la probabilidad de detecci\'on por las c\'amaras trampa
($p^{\mbox{det}}$) de cada uno de los individuos de la poblaci\'on.  A
partir de estos valores podemos generar datos de noches de trampeo, suponiendo
que el n\'umero de individuos detectados en una noche es una variable
aleatoria de una distribuci\'on binomial con par\'ametros $n=N_o$ y
$p=p^{\mbox{det}}$.

Por ejemplo, con una poblaci\'on inicial de $N_o=$\py{datosJson["No"]["data"]}
individuos y una probabilidad de detecci\'on $p^{\mbox{det}}=$
\py{datosJson["q"]["data"]}, se simularon \py{datosJson["nDatos"]["data"]}
noches de trampeo (se generaron las observaciones).
As\'i, en las primeras 5 noches de trampeo el n\'umero de observaciones fueron
$\vec{n}=($\py{datosJson["datos"]["data"][0]}, \py{datosJson["datos"]["data"][1]},
\py{datosJson["datos"]["data"][2]}, \py{datosJson["datos"]["data"][3]},
\py{datosJson["datos"]["data"][4]}); es decir,
la primera noche se observaron \py{datosJson["datos"]["data"][0]} individuos, la
segunda noche \py{datosJson["datos"]["data"][1]} individuos y la quinta
\py{datosJson["datos"]["data"][4]} individuos.

% En esta sección se hace una pequeña descripción de los resultados utilizando
% dos gráficas como herramienta.
\subsection*{Resultados}
Se hizo un muestreo de Gibbs de tama\~no 5000, para obtener las FDP
posteriores para los par\'ametros $p^{\mbox{det}}$ y $N_o$. Los resultados
los presentamos en las figuras \ref{fproCap} y \ref{ftamIni}. Los muestreos para
$p^{\mbox{det}}$ se hacen a partir de funciones betas de distribuci\'on de
probabilidad:
$$\beta(x;a,b)=\left(\frac{\Gamma(a+b)}{\Gamma(a)\Gamma(b)}p^{a-1}(1-p)^{b-1}\right);$$
y para $N_o$ se consider\'o distribuciones uniformes. Las funciones beta
multiplicada por funciones binomiales son funciones beta. Es por eso que
decidimos utilizar funciones beta.


La figura \ref{fproCap} muestra el histograma de la FDP
de los valores para el par\'ametro $p^{\mbox{det}}$. Se observa que
el m\'aximo de repeticiones corresponde a la probabilidad $p^{\mbox{det}}=$
\py{eje_x[eje_y.index(max(eje_y[1:-1]))]}.


La figura \ref{ftamIni} muestra el histograma de la FDP
 de los valores para el par\'ametro $N_o$. Se observa que
el valor m\'inimo de la poblaci\'on inicial (el eje horizontal) es mayor a
\py{max(datosJson["datos"]["data"])} individuos, que corresponde al valor
m\'aximo de los datos sint\'eticos de observaciones ($n_i$) en las
\py{datosJson["nDatos"]["data"]} noches de trampeo simuladas.

% Aquí va figura `resultados\probabilidad_captura.png` que muestra el histograma
% de la probabilidad de captura de los blancos. Los datos son generados por una
% muestra de Gibbs que se realiza con el programa `python\bates.py`.
\begin{figure}[h]
\centering
\includegraphics[width=120mm]{../resultados/probabilidad_captura.png}
\caption{Histograma de la muestra de probabilidades de capturas
$p^{\mbox{det}}$, generadas a partir del m\'etodo de Gibbs.}
\label{fproCap}
\end{figure}
% Aquí va figura `resultados\tamagno_inicial.png` que muestra el histograma
% del tamaño inicial de una población de blancos. Los datos son generados por
% una muestra de Gibbs que se realiza con el programa `python\bates.py`.
\begin{figure}[h]
\centering
\includegraphics[width=120mm]{../resultados/tamagno_inicial.png}
\caption{Histograma de los tama\~nos iniciales
$N_o$, generados a partir del m\'etodo de Gibbs.}
\label{ftamIni}
\end{figure}

% En esta sección se hace una pequeña discusión de los resultados mostrados en
% las dos gráficas.
\subsection*{Discusi\'on}
Las columnas en blanco en los histogramas de las figuras \ref{fproCap} y
\ref{ftamIni} podr\'ian deberse a la necesidad de un tama\~no de muestra m\'as
grande. Se puede observar que los histogramas ya tienen la forma de las
distribuciones de probabilidad {\it posteriores}.

La propuesta de una distribuci\'on uniforme para $N_o$ no parece la m\'as
adecuada. No se puede sacar mucha informaci\'on de ella ya que cualquier
tama\~no inicial de poblaci\'on es igual de probable. Una distribuci\'on
multinomial podr\'ia proporcionar mayor informaci\'on, ya que permite que se
tengan probabilidades distintas para distintos tama\~nos de poblaci\'on.

Los datos con los que se gener\'o la figura \ref{fproCap} servir\'an para
obtener la FDP de variable aleatoria
$p^{\mbox{det}}$. Esa FDP se usar\'a para calcular
los intervalos de confianza para $p^{\mbox{det}}$. Debemos recordar que la EB
considera a los par\'ametros $p^{\mbox{det}}$ y $N_o$ como variables aleatorias,
por lo que el valor m\'aximo de
$p^{\mbox{det}}=$\py{eje_x[eje_y.index(max(eje_y[1:-1]))]} no deber ser
considerado como el valor buscado, sino solo informaci\'on descriptiva de la FDP
a la que pertenece.

\subsection*{Conclusi\'on}
El reporte presenta la metodolog\'ia para obtener la distribuci\'on de
probabilidad de algunos par\'ametros de inter\'es para la estimaci\'on del
tama\~no inicial de una poblaci\'on. Los resultados los obtuvimos
considerando situaciones en las que no hay extracci\'on de individuos.

Obtuvimos histogramas para la probabilidad de detecci\'on de individuos y el
tama\~no inicial de la poblaci\'on a partir de muestreos de Gibbs. Creemos que
se obtendr\'an mejores resultados si el tama\~no de muestra crece. Adem\'as,
un cambio en el modelo podr\'ia proporcionar m\'as informaci\'on; por ejemplo,
que el tama\~no de la poblaci\'on siga una distribuci\'on multinomial.

Hasta el momento, no hemos calculado las distribuciones de probabilidad de
las variables del $N_o$ y $p^{\mbox{det}}$, pero son necesarias para obtener
de los intervalos de confianza y poder hacer una mejor evaluaci\'on del modelo y
los resultados.

\newpage

\bibliography{referencias}
\end{document}
